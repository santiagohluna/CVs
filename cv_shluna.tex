\documentclass[12pt,a4paper]{article}
\usepackage[utf8]{inputenc}
\usepackage[spanish]{babel}
\usepackage[T1]{fontenc}
\usepackage{charter}
\usepackage{longtable}
\usepackage[left=1cm,right=1cm,top=2cm,bottom=2cm]{geometry}
\usepackage{hyperref}

\hypersetup{
   linktocpage=true,
    colorlinks=true,
    linkcolor=blue,
    citecolor=blue,
    filecolor=blue,      
    urlcolor=blue
}

\title{Santiago Hernán Luna}
\author{Currículum Vítae}
\date{ }

\renewcommand{\arraystretch}{1.5}

\begin{document}

\maketitle

\begin{longtable}[t]{r p{12cm}}

%         & \textbf{Datos personales} \\
%  \cline{2-2}
%  \textbf{DNI nro.}                    & 31710472 \\
%  \textbf{Fecha de nacimiento}         & 30 de agosto de 1985. \\
%  \textbf{Lugar de nacimiento}         & Rufino, Provincia de Santa Fe. \\
%  \textbf{Nacionalidad}                & Argentino. \\
%  \textbf{Estado civil}                & Soltero, sin hijos. \\

        & \\
        & \textbf{Datos de contacto} \\
 \cline{2-2}
 \textbf{Domicilio}                   & Villa Crespo, Ciudad Autónoma de Buenos Aires. \\
 \textbf{Teléfono celular}            & (+54) 3382 671295 \\
 \textbf{E-mail} & \href{mailto:santiagohluna@gmail.com}{santiagohluna@gmail.com} \\

 & \\
 & \textbf{Datos Laborales} \\
\cline{2-2}
\textbf{Lugar de trabajo}   & Secretaría de Investigación, Universidad Nacional de Hurlingham. \\
\textbf{Dirección}          & Teniente Origone 151, Villa Tesei, Hurligham, Provincia de Buenos Aires, Argentina. \\
\textbf{C. P.}              & B1688AXC \\     
\textbf{Teléfono}           & (+54) 11 2078 5200 \\
\textbf{Correo electrónico} & \href{mailto:santiago.luna@unahur.edu.ar}{santiago.luna@unahur.edu.ar} \\
\textbf{Posición}           & Docente e investigador. \\

& \\
& \textbf{Cargos docentes} \\
\cline{2-2}
\textbf{10-2023 -- Actualidad} & Docente adjunto. Secretaría de investigación, Universidad Nacional de Hurlingham. \\
\textbf{05-2022 -- Actualidad} & Docente auxiliar (Jefe de trabajos prácticos). Instituto de Tecnología e Ingeniería, Universidad Nacional de Hurlingham. \\
\textbf{03-2021 -- 05-2022} & Docente auxiliar (Ayudante de primera). Instituto de Tecnología e Ingeniería, Universidad Nacional de Hurlingham. \\

& \\
& \textbf{Empleos académicos anteriores} \\ 
\cline{2-2}
\textbf{2020 -- 2023} & Becario posdoctoral (CONICET) en el Instituto de Estudios Andinos ``Don Pablo Groeber'' (Universidad de Buenos Aires -- CONICET). \\
\textbf{2015 -- 2020} & Becario doctoral (CONICET) en el Instituto de Astronomía y Física del espacio (CONICET--Universidad de Buenos Aires). \\

 & \\
 & \textbf{Formación académica} \\
\cline{2-2}
\textbf{2015 -- 2020} & Doctorado en Física. Facultad de Ciencias Exactas, Ingeniería y Agrimensura. Universidad Nacional de Rosario. \\
\textbf{2008 -- 2015} & Licenciatura en Física. Facultad de Ciencias Exactas, Ingeniería y Agrimensura. Universidad Nacional de Rosario. \\
%\textbf{2004 -- 2008} & Ingeniería Mecánica. Facultad Regional Rosario de la Universidad Tecnológica Nacional. Hasta tercer año de la carrera (Incompleto). \\

& \\
& \textbf{Experticia} \\
\cline{2-2}
\textbf{Descripción} & Me especializo en el estudio de la dinámica rotacional y orbital de sistemas planetarios (Solar y Extrasolares) y sistemas de satélites debido a la presencia de varios componentes (problema de N cuerpos), mareas sólidas y aquellas originadas por una distribución inhomogénea y permanente de masa en el cuerpo central. También, me dedico al estudio de la interrelación entre evolución térmica y dinámica de sistemas planetarios. Básicamente, mi metodología de trabajo es de perfil teórica, esto es, formular modelos matemáticos que luego se implementan en códigos computacionales para llevar a cabo simulaciones numéricas, por lo que tengo experiencia en programación y en el manejo de grandes volúmenes de datos. \\
\textbf{Palabras clave} & Mecánica clásica --- Mecánica celeste --- Dinámica orbital y rotacional --- Teoría de mareas sólidas --- Reología --- Sistemas planetarios --- Interacción entre estrellas y planetas --- Interacción entre planetas y satélites. -- Geofísica -- Estructura interna -- Evolución térmica y dinámica acopladas \\

  & \\
  & \textbf{Cursos académicos} \\
 \cline{2-2}
 \textbf{2019} & Mecánica Clásica Avanzada (asignatura de posgrado, cursada y aprobada). Facultad de Ciencias Exactas, Ingeniería y Agrimensura de la Universidad Nacional de Rosario. \\
 \textbf{2019} & Procesamiento de Datos (asignatura de posgrado, cursada y aprobada). Facultad de Ciencias Exactas, Ingeniería y Agrimensura de la Universidad Nacional de Rosario. \\
 \textbf{2019} & Física del interior terrestre (asignatura de la Licenciatura en Geofísica, cursada en el año 2015 y aprobada por examen final). Facultad de Ciencias Astronómicas y Geofísicas de la Universidad Nacional de La Plata. \\
 \textbf{2016} & Mareas Terrestres (seminario de posgrado, cursado y aprobado). Facultad de Ciencias Astronómicas y Geofísicas de la Universidad Nacional de La Plata. \\
 \textbf{2015} & Astrofísica computacional (asignatura de posgrado, cursada y aprobada). Facultad de Ciencias Exactas, Ingeniería y Agrimensura de la Universidad Nacional de Rosario. \\
 \textbf{2014} & Física computacional y Fundamentos de la investigación (unidades curriculares electivas del 5º año de la Licenciatura en Física, cursadas y aprobadas).  Facultad de Ciencias Exactas, Ingeniería y Agrimensura de la Universidad Nacional de Rosario. \\
 \textbf{2007} & Curso de Astronomía general (aprobado). Planetario Municipal de Rosario. \\

%  \pagebreak
%  & \\
%  & \textbf{Cursos de formación profesional} \\
% \cline{2-2}
% \textbf{2022} & Análisis y diseño de casos de prueba. Certificados de \href{https://1drv.ms/b/s!AgXxKbaw9rfPiARWpusuUxhE0wli?e=fpeg86}{asistencia} y \href{https://1drv.ms/b/s!AgXxKbaw9rfPiAJQ5pvh-rHR5E9P?e=rqxedq}{aprobación} otorgados por EducaciónIT. \\
%               & Software Tester QA Manual. Certificados de \href{https://1drv.ms/b/s!AgXxKbaw9rfPinl2OYyScnj9cT-2?e=3zIFgY}{asistencia} y \href{https://1drv.ms/b/s!AgXxKbaw9rfPh0PhetsnIUSpbF12?e=pu1IeQ}{aprobación} otorgados por EducaciónIT. \\
%               & JAVA para no programadores. Certificados de \href{https://1drv.ms/b/s!AgXxKbaw9rfPiT17pM40SCem4Bz5?e=24jo5G}{asistencia} y \href{https://1drv.ms/b/s!AgXxKbaw9rfPil5vd1FMXUgKGw6Y?e=23fAvz}{aprobación} otorgados por EducaciónIT. \\
%               & Desarrollo web con HTML. Certificados de \href{https://1drv.ms/b/s!AgXxKbaw9rfPiRQvFhaJl9l61Lcc?e=EQmXAr}{asistencia} y \href{https://1drv.ms/b/s!AgXxKbaw9rfPixilMsmDXOfCizDJ?e=bclJY0}{aprobación} otorgados por EducaciónIT. \\
%               & Introducción a bases de datos y SQL. Certificado de \href{https://1drv.ms/b/s!AgXxKbaw9rfPikmF4wx1J_vxqFGi?e=LCSQZh}{asistencia} otorgado por EducaciónIT. \\
%               & Introducción al paradigma de objetos. Certificados de \href{https://1drv.ms/b/s!AgXxKbaw9rfPiX-Dr1cu1LTxGIEa?e=qyfRm5}{asistencia} y \href{https://1drv.ms/b/s!AgXxKbaw9rfPixS5KpEkcy02sFvT?e=gy3wY9}{aprobación} otorgados por EducaciónIT. \\
%               & Introducción a DevOps. Certificados de \href{https://1drv.ms/b/s!AgXxKbaw9rfPiwIOPv-l5QB5ywGL?e=wUloD9}{asistencia} y \href{https://1drv.ms/b/s!AgXxKbaw9rfPiyqJemrAly3msqm-?e=s3LDTx}{aprobación} otorgados por EducaciónIT. \\
%               & Protocolo HTTPS. Certificado de \href{https://1drv.ms/b/s!AgXxKbaw9rfPiznTwZw5hsxe72iq?e=upVLIV}{aprobación} otorgados por EducaciónIT. \\
%              & Scrum: Fundamentos. Certificados de \href{}{asistencia} y \href{https://1drv.ms/b/s!AgXxKbaw9rfPi2BVNF80eONCtgcd?e=8e7EPd}{aprobación} otorgados por EducaciónIT. \\

  & \\
  & \textbf{Idiomas} \\
 \cline{2-2}
 \textbf{Español} & Nativo. \\
 \textbf{Inglés}  & Avanzado. \\
 \textbf{Francés} & Intermedio (certificación \textsc{delf a1}). \\
 \textbf{Alemán}  & Básico. \\

  & \\
  & \textbf{Informática} \\
 \cline{2-2}
 \parbox[t]{3cm}{\textbf{Lenguajes de programación}} & \textsc{fortran} y Python. Específicamente, su aplicación al modelado de sistemas en Física. \\
 \parbox[t]{3cm}{\textbf{Sistemas \\ operativos}} & Linux (Ubuntu) y Windows. \\
 \parbox[t]{3cm}{\textbf{Aplicaciones ofimáticas}} & LibreOffice y Microsoft Office. Además utilizo con fluidez el sistema de composición de documentos \LaTeX. \\
 \textbf{Cálculo simbólico} & Máxima, Sympy y Mathematica. \\
 \textbf{Gráficos} & Gnuplot y Matplotlib. \\
 \textbf{Otros} & Uso fluido de Git, GitHub y Rsync. \\

\pagebreak

 & \\
 & \textbf{Trabajos publicados} \\
\cline{2-2}
\textbf{2024} & Luna, S. H., Spagnuolo, M. G. y Navone, H. D. ``Tidal heating in the history of the Earth: wasn't it really important?''. En preparación. \\
              & Spagnuolo, M.G., Mantegazza, M., \textbf{Luna, S.H.} ``Illumination Conditions at Mars and Their Relationships with Ice-Driven Morphology'' en \textit{Latin American Geomorphology}. Coronato, A., Alves, G.B., eds. The Latin American Studies Book Series. Springer, Cham. Disponible en \url{https://doi.org/10.1007/978-3-031-55178-9_7}. \\
\textbf{2023} & \textbf{Luna, S. H.}, Spagnuolo, M. G. y Navone, H. D. ``Estudio de diferentes escenarios en el modelado de la influencia de la interacción de mareas en la evolución térmica del interior terrestre'' en  
\textit{Boletín de la Asociación Argentina de Astronomía}. R.D. Rohrmann et al., eds. La Plata, Argentina. Vol. 64, p. 32-34. Disponible en \url{http://www.astronomiaargentina.org.ar/b64/2023BAAA...64...32L.pdf}. \\
\textbf{2022} & \textbf{Luna, S. H.}, Spagnuolo, M. G. y Navone, H. D. “Influencia de la temperatura interna de la Tierra y de la Luna en la evolución dinámica del sistema Tierra-Luna” en \textit{Boletín de la Asociación Argentina de Astronomía}, vol. 63, pp. 42-44. Rohrmann C. H. et al., eds. La Plata, Argentina. Disponible en \url{http://www.astronomiaargentina.org.ar/b63/2022BAAA...63...42L.pdf}. \\
              & Chinellato, L., \textbf{Luna, S. H.}, Pera, M. S., Perren, G. I., Menchón,
              R., Spagnuolo, M. G., Navone, H. D. “Clasificación de exoplanetas: desarrollo de una estrategia didáctica para abordar la construcción de modelos observacionales en Física Educativa”. \textit{Latin America Journal of Physics Education}, Vol. 16, No. 4, Dec., 2022. Disponible en \url{http://www.lajpe.org/dec22/16_4_07.pdf}. \\
\textbf{2021} & \textbf{Luna, S. H.}, Spagnuolo, M. G. y Navone, H. D. ``Evaluación del impacto de la interacción de mareas en la evolución térmica del manto terrestre'' en \textit{Boletín de la Asociación Argentina de Astronomía}, vol. 62, pp. 53-55. Vazquez A. M. \textit{et al}., eds. La Plata, Argentina. Disponible en \url{http://www.astronomiaargentina.org.ar/b62/2021BAAA...62...53L.pdf}. \\
              & Niell, L., \textbf{Luna, S. H.}, Fourty, A. y Navone, H. D. ``El fenómeno de las mareas: consideraciones sobre su abordaje enlos libros universitarios de física''. Revista De Enseñanza De La Física, vol. 33, pp. 495-502. Disponible en \url{https://revistas.unc.edu.ar/index.php/revistaEF/article/view/35593}. \\
\textbf{2020} & \textbf{Luna, S. H.}, Navone, H. D. y Melita, M. D. ``The dynamical evolution of close-in binary systems formed by a super-Earth and its host star -- Case of the Kepler-21 system''. Astronomy \& Astrophysics, vol. 641 : A109. Resumen disponible en \url{https://www.aanda.org/articles/aa/abs/2020/09/aa36551-19/aa36551-19.html}.\\
\textbf{2019} & \textbf{Luna, S. H.} y Navone, H. D. ``La anomalía verdadera en función del tiempo como solución de un problema de valor inicial'' en \textit{Ciencia y Tecnología 2018: divulgación de la producción científica y tecnológica de la UNR}, pp. 534-540. Orellano, E. \textit{et al}., coords. Rosario, Argentina. Disponible en \url{https://www.dropbox.com/s/ynsc8c50ep54ks0/Luna_LA%20ANOMALIA%20VERDADERA_Ampliado.pdf?dl=0}. \\
\textbf{2018} & \textbf{Luna, S. H.}, Melita, M. D. y Navone, H. D. ``Estudio de la evolución orbital de Fobos debido a la interacción de mareas y su relación con las propiedades físicas de Marte'' en \textit{Boletín de la Asociación Argentina de Astronomía}, vol. 60, pp. 265-267. Benaglia P. \textit{et al}., eds. La Plata, Argentina. Disponible en \url{http://articles.adsabs.harvard.edu/pdf/2018BAAA...60..265L}. \\
              & \textbf{Luna, S. H.}, Melita, M. D. y Navone, H. D. ``Origen y evolución orbital de Fobos: Exploración de una hipótesis de captura'' en \textit{Boletín de la Asociación Argentina de Astronomía}, vol. 60, pp. 268-270. Benaglia P. \textit{et al}., eds. La Plata, Argentina. Disponible en \url{http://articles.adsabs.harvard.edu/pdf/2018BAAA...60..268L}. \\
              & \textbf{Luna, S. H.} y Navone, H. D. ``Expansión del potencial perturbador en función de los elementos orbitales y su aplicación al estudio de la interacción de mareas'' en \textit{Ciencia y Tecnología 2017: divulgación de la producción científica y tecnológica de la UNR}, pp. 620-627. Orellano, E. \textit{et al}., coords. Rosario, Argentina. Disponible en \url{https://www.dropbox.com/s/ypqlwepdituxkw4/Luna_Expansion-Potencial-Gravitatorio_Ampliado_secyt2017.pdf?dl=0}.\\
 \textbf{2017} & \textbf{Luna, S. H.}, Menchón, R., Perrén, G., Manuel, L., Navone, H. ``El campo de la Astrofísica en la formación inicial de profesores de educación secundaria en Física: Análisis cualitativo de diseños curriculares jurisdiccionales'' en \textit{Ciencia y Tecnología 2016: divulgación de la producción científica y tecnológica de la UNR}, pp. 705-712. Orellano, E. \textit{et al}., coords. Rosario, Argentina. Disponible en \url{https://www.dropbox.com/s/fl9z8vd447x1oh6/secyt2016_shluna_El_campo_de_la_Astrofisica.pdf?dl=0}. \\
               & \textbf{Luna, S. H.}, Navone, H. D. y Melita, M. D. ``Evolución rotacional y orbital debida a la interacción de mareas en sistemas exoplanetarios observados: Análisis comparativo de diversas metodologías de cálculo'' en \textit{Ciencia y Tecnología 2016: divulgación de la producción científica y tecnológica de la UNR}, pp. 713-720. Orellano, E. \textit{et al}., coords. Rosario, Argentina. Disponible en \url{https://www.dropbox.com/s/sq83cynbknt102c/Luna_Evolucion%20rotacional%20y%20orbital_Ampliado.pdf?dl=0}. \\
               & Menchón, R., \textbf{Luna, S. H.}, Manuel, L., Navone, H. ``La Mecánica clásica como campo problemático en los diseños curriculares jurisdiccionales de profesorados en Física: Estudio exploratorio de casos'' en \textit{Ciencia y Tecnología 2016: divulgación de la producción científica y tecnológica de la UNR}, pp. 729-737. Orellano, E. \textit{et al}., coords. Rosario, Argentina. Disponible en \url{https://www.dropbox.com/s/3mrszhyr4a7vsq4/secyt2016_menchon_La_Mecanica_Clasica.pdf?dl=0}. \\
\textbf{2016} & \textbf{Luna, S. H.}, Melita, M. D. y Navone, H. D. ``Estudio de la interacción de mareas en sistemas exoplanetarios observados: Estimación de las probabilidades de captura en resonancias spín-órbita'' en \textit{Boletín de la Asociación Argentina de Astronomía}, vol. 58, pp. 310-312. Benaglia P. \textit{et al}., eds. La Plata, Argentina. Disponible en \url{http://articles.adsabs.harvard.edu/pdf/2016BAAA...58..310L}. \\
              & \textbf{Luna, S. H.}, Navone, H. D. y Melita, M. D. ``Sistema exoplanetario HD 154088: Estimación de las probabilidades de captura en las resonancias 3:2, 2:1 y 5:2'' en \textit{Ciencia y Tecnología 2016: divulgación de la producción científica y tecnológica de la UNR}, pp. 917-924. Orellano, E. \textit{et al}., coords. Rosario, Argentina. Disponible en \url{https://www.dropbox.com/s/9betljabrpjp23z/Luna_Sistema_Exoplanetario_HD154088.pdf?dl=0}. \\

%  & \\
%  & \textbf{Trabajos en preparación} \\
% \cline{2-2}
% \textbf{2019} & Luna, S. H., Melita, M.D. y Navone, H. D. ``Dynamical evolution of close-in super-earths tidally interacting with its host star near spin-orbit resonances. The case of Kepler-21 system''. Versión preliminar disponible en \href{http://arxiv.org/abs/1907.10575}{arXiv}. \\

  & \\
  & \textbf{Participación en reuniones, talleres y jornadas} \\
 \cline{2-2}
 \textbf{2022} & 64ª Reunión anual de la Asociación Argentina de Astronomía. Exposición mural. Buenos Aires, Argentina. \\
 \textbf{2021} & 63ª Reunión anual de la Asociación Argentina de Astronomía. Exposición mural. Córdoba, Argentina. \\
 \textbf{2020} & 62ª Reunión anual de la Asociación Argentina de Astronomía. Exposición mural. Rosario, Argentina. \\
               & Jornadas de Ciencia, Tecnología e Innovación. Exposición mural. Rosario, Argentina. \\
 \textbf{2018} & IX Taller de Ciencias Planetarias. Exposición oral y mural. La Plata, Argentina. \\
               & XII Jornada de Ciencia y Tecnología. Exposición mural. Rosario, Argentina. \\
 \textbf{2017} & 60ª Reunión anual de la Asociación Argentina de Astronomía. Exposición oral y mural. Malargüe, Argentina. \\
               & XI Jornada de Ciencia y Tecnología. Exposición mural. Rosario, Argentina. \\
 \textbf{2016} & VIII Taller de Ciencias Planetarias. Exposición mural. Porto Alegre, Brasil. \\
               & X Jornada de Ciencia y Tecnología. Exposición mural. Rosario, Argentina. \\
 \textbf{2015} & 58ª Reunión anual de la Asociación Argentina de Astronomía. Exposición mural. La Plata, Argentina. \\
               & IX Jornada de Ciencia y Tecnología. Exposición mural. Rosario, Argentina. \\

  & \\
  & \textbf{Participación en proyectos de investigación} \\
 \cline{2-2}
  &  ``Análisis de sistemas astrofísicos con datos Gaia: parametrización de cúmulos abiertos y objetos conexos''. Proyecto de Investigación acreditado en la Universidad Nacional de Rosario. Código: 80020210300042UR. Vigencia: 2022-2025. \\
  & ``Construcción de referentes teóricos y diseño de estrategias didácticas que posibiliten la inserción de problemáticas transversales en la formación de educadorxs en Física''. Proyecto de Investigación acreditado en la UNR. Código: 80020210200086UR. Vigencia: 2022-2025. \\
  &  Universidad Nacional de Rosario (2015-2016) – Código: ING 430 - Proyecto: Diseño, implementación y evaluación de estrategias didácticas en Educación Ambiental desde una perspectiva interdisciplinar, integradora y compleja. Director: Hugo D. Navone. \\
  &  \textsc{conicet - pip} 11220150100699CO (2015-2017) - Proyecto: Estudios teóricos en Ciencia Planetaria: Cuerpos Menores del Sistema Solar y Sistemas Extra-Solares. Director: Mario Melita. \\
  & \textsc{anpcyt - pict 1144-13} (2015-2019) - Proyecto: Desarrollo de un Observatorio Robótico Antártico Argentino. Director: Mario Melita. \\
  & Universidad Nacional de Rosario (2017-2020) – Código: ING 545 - Proyecto: Diseño y desarrollo curricular de estrategias didácticas de carácter transversal e integrador destinadas a enriquecer la formación inicial y/o permanente de educadores en Física. Director: Hugo D. Navone. \\
 & \\
 & \textbf{Membresía en asociaciones científicas} \\
\cline{2-2}
\textbf{2021 -- Actualidad} & Miembro de la Asociación de Profesores de Física de Argentina. \\
\textbf{2021 -- Actualidad} & Miembro junior de la Unión Astronómica Internacional. \\
\textbf{2015 -- Actualidad} & Socio profesional de la Asociación Argentina de Astronomía. \\

\end{longtable}

\end{document}
